\documentclass[a4paper,12pt]{article}
  
% Paquetes fundamentales para tu TFG
\usepackage[utf8]{inputenc}
\usepackage[spanish]{babel}
\usepackage{amsmath}  % Matemáticas avanzadas
\usepackage{amssymb}  % Símbolos (R, C, etc.)
\usepackage{physics}  % Derivadas fáciles (\dd{x}{t})

\title{Prueba de Entorno: TFG Física Aceleradores}
\author{Jaime}
\date{\today}

\begin{document}

\maketitle

\section{Introducción}
Si estás leyendo esto en el PDF de la derecha, \textbf{tu entorno funciona perfectamente}.

\section{Ecuación de Hill}
El objetivo del TFG es extender la ecuación lineal:
\begin{equation}
    \dv[2]{x}{s} + K(s)x(s) = 0
\end{equation}

A su versión no lineal con términos de perturbación:
\begin{equation}
    x'' + K(s)x + \underbrace{\frac{1}{2!}m x^2}_{\text{Sextupolo}} + \dots = 0
\end{equation}

\end{document}